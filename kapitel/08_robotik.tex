% !TeX root = summary.tex

\section{Robotik\index{Robotik}}

\subsection{Einführung}

Aufbau eines kognitiven Systems:

% Bild Folie 26
\mypic{8}{kognitives_system}

Aktion:
\begin{itemize}
\item Ausführung einer geplanten Handlung
\item Ergebnis des Verarbeitungsschritts Perzeption $\to$ Kognition $\to$ Aktion
\item Computersystem mit aktorischer Komponente: "{}Roboter"{}
\begin{itemize}
\item \textsl{robota [poln.]}: Fronarbeit
\item zuerst verwendet in dem futur. Theaterstück "{}Rossum's Universal Robots"{} von Karel Capek
\end{itemize}
\end{itemize}

Typen von Robotersystemen:
\begin{itemize}
\item Industrieroboter, automatische Fertigungssysteme, Roboterarme
\begin{itemize}
\item stationär, hochspezialisiert, allgemein verfügbar
\item Anwendungen: Schweißen, Schneiden, Fräsen, Lackieren, Bestücken, \dots
\end{itemize}
\item Autonome mobile Roboter
\begin{itemize}
\item mobil, auf Transportaufgaben ausgerichtet, noch recht selten
\item Anwendungen: Lagerhaltung, Waren-, Nachrichtentransport, Führungen, \dots
\end{itemize}
\item Humanoide Assistenzsysteme
\begin{itemize}
\item mobil, allgemeinbefähigt, Gegenstand aktueller Forschung
\item Anwendungen: Fertigungs-, Haushaltsassistenz, Pflege, Begleitung, Schutz, \dots \\ $\to$ "{}Alleskönner"{}
\end{itemize}
\end{itemize}

Probleme der Robotik:
\begin{itemize}
\item Kinematik:
\begin{itemize}
\item Lehre der Bewegung von Körpern
\item Wohin bewegt sich der Roboter, wenn er Gelenk $a$ auf $45\degree$ einstellt?
\item Wie müssen die Gelenke eingestellt werden, damit sich der Roboter von $A$ nach $B$ bewegt?
\item Ist diese Bewegung möglich?
\end{itemize}
\item Dynamik
\begin{itemize}
\item Lehre von Geschwindigkeiten, Kräften, Momenten
\item Wieviel Kraft muss der Roboter aufwenden, um Gelenk $a$ auf $45\degree$ einzustellen?
\item Welche Kräfte entstehen bei Bewegung $A$ nach $B$?
\item Geht dabei auch nichts kaputt?
\end{itemize}
\item Konfigurations-, Hindernis-, Freiraum, Kollisionsvermeidung, \dots
\end{itemize}

\subsection{Robotermodellierung}

Zum Lösen der Probleme der Robotik Kinematik, Dynamik, Bahnplanung stellt man verschiedene Modelle auf den Roboter auf:

% Bild Folie 31
\mypic{3}{roboter_modell}

\subsubsection*{Geometrische Modellierung}

\textbf{Geometrisches Modell}
\begin{itemize}
\item dient als Grundlage zur Berechnung der Bewegung
\item wird benutzt zur Ermittlung der wirkenden Kräfte und Momente
\item stellt den Körper graphisch dar
\item ist Ausgangspunkt der Abstandsmessung und Kollisionserkennung
\end{itemize}
Komplexitätsstufen des geometrischen Modells:
\begin{enumerate}
\item Bounding-Box-Modell\index{Bounding-Box-Modell}
\begin{itemize}
\item die Körper werden durch einhüllende Quader dargestellt
\item wird in den ersten Schritten der Kollisiosvermeidung benutzt
\end{itemize}
\item Kantenmodell\index{Kantenmodell}
\begin{itemize}
\item die Körper werden durch Polygonzüge (Kanten) dargestellt
\item wird zur schnellen Visualisierung (Rechenleistung für Grafik) benutzt
\end{itemize}
\item Volumenmodell\index{Volumenmodell}
\begin{itemize}
\item die Körper werden genau dargestellt
\item wird für die Ermittlung der genauen Werte der Kollisionserkennung benutzt
\end{itemize}
\end{enumerate}

\subsubsection*{Kinematische Modellierung}

\begin{description}
\item[Kinematisches Modell] ermittelt mit Hilfe des geometrischen Modells die Stellung (Position und Orientierung) von Körpern im Raum. \\ Eine \textbf{kinematische Kette} wird von mehreren Körpern gebildet, die durch Gelenke kinematisch verbunden sind (z.B. Roboterarm)
\end{description}

Zweck des kinematischen Modells:
\begin{itemize}
\item Bestimmung des Zusammenhangs zwischen Gelenkwerten und Stellungen
\item Erreichbarkeitsanalyse: Welche Punkte sind anfahrbar, ohne dass innere oder äußere Kollisionen auftreten?
\end{itemize}
Aufstellung des kinematischen Modells erfolgt iterativ. Jedes Armelement entspricht einem starren Körper. (Bsp.: Puma 260, 6-achsiger Roboter, Basis und 6 Armelemente) \\
Zu jedem Körper existiert ein Objekt-Koordinatensystem
\begin{center}
\begin{tabular}{l@{\,\,\,}l}
Roboterbasis & OKS$_{Basis}$ \\
Armelement 1 & OKS$_{Arm1}$ \\
\vdots & \vdots \\
Armelement 6 & OKS$_{Arm6}$
\end{tabular}
\end{center}
Koordinatensysteme werden urch homogene Transformationen miteinander verknüpft.

\subsubsection*{Wdh. homogene Koordinaten}

\newcommand{\myrot}[1]{rot_{#1}(\alpha)}

$4 \times 4$-Matrizen, die geschlossen Rotation und Translation ausdrücken.
\begin{eqnarray*}
p + Rot(\alpha , p') &=& \left( \begin{array}{cccc} \myrot{0,0} & \myrot{0,1} & \myrot{0,2} & 0 \\ \myrot{1,0} & \myrot{1,1} & \myrot{1,2} & 0 \\ \myrot{2,0} & \myrot{2,1} & \myrot{2,2} & 0 \\ 0 & 0 & 0 & 1 \end{array} \right) \left( \begin{array}{cccc} 1 & 0 & 0 & x' \\ 0 & 1 & 0 & y' \\ 0 & 0 & 1 & z' \\ 0 & 0 & 0 & 1 \end{array} \right) \left( \begin{array}{c} x \\ y \\ z \\ 1 \end{array} \right) \\
&=& \left( \begin{array}{cccc} \myrot{0,0} & \myrot{0,1} & \myrot{0,2} & x' \\ \myrot{1,0} & \myrot{1,1} & \myrot{1,2} & y' \\ \myrot{2,0} & \myrot{2,1} & \myrot{2,2} & z' \\ 0 & 0 & 0 & 1 \end{array} \right)\left( \begin{array}{c} x \\ y \\ z \\ 1 \end{array} \right)
\end{eqnarray*}

\subsubsection*{Denavit-Hartenberg-Transformation}

Die Transformation vom OKS des $i$-ten Armelements auf das OKS des $(i-1)$-ten Armelements erfolgt nach der \textbf{Denavit-Hartenberg (DH)-Konvention}\index{Denavit-Hartenberg-Konvention}:
\begin{enumerate}
\item[(a)] die Koordinatensysteme liegen in den Bewegungsachsen
\item[(b)] die $z_{i-1}$-Achse liegt entlang der Bewegungsachse des $i$-ten Armelements
\item[(c)] die $x_i$-Achse steht senkrecht zur $z_{i-1}$-Achse und zeigt von ihr weg
\end{enumerate}
($i \in \{ Basis, 1 , \dots , n \}$) \\[0,1cm]
\textbf{DH-Parameter} \\
Transformation von OKS$_{i-1}$ auf OKS$_i$:
\begin{enumerate}
\item Eine Rotation $\theta_i$ um die $z_{i-1}$-Achse, damit die $x_{i-1}$-Achse parallel zu der $x_i$-Achse liegt (wegen Voraussetzung (c)).
$$R_{z_{i-1}}(\theta_i) = \left[ \begin{array}{cccc} \cos(\theta_i) & - \sin(\theta_i) & 0 & 0 \\ \sin(\theta_i) & \cos(\theta_i) & 0 & 0 \\ 0 & 0 & 1 & 0 \\ 0 & 0 & 0 & 1 \end{array} \right]$$
\item Eine Translation $s_i$ entlang der $z_{i-1}$-Achse zu dem Punkt, wo sich $z_{i-1}$ und $x_i$ schneiden (wegen Voraussetzungen (b), (c)).
$$T_{z_{i-1}}(s_i) = \left[ \begin{array}{cccc} 1 & 0 & 0 & 0 \\ 0 & 1 & 0 & 0 \\ 0 & 0 & 1 & s_i \\ 0 & 0 & 0 & 1 \end{array} \right]$$
\item Eine Translation $a_i$ entlang der $x_i$-Achse um die Ursprünge in Deckung zu bringen (wegen Voraussetzung (c)).
$$T_{x_i}(a_i) = \left[ \begin{array}{cccc} 1 & 0 & 0 & a_i \\ 0 & 1 & 0 & 0 \\ 0 & 0 & 1 & 0 \\ 0 & 0 & 0 & 1 \end{array} \right]$$
\item Eine Rotation $\alpha_i$ um die $x_i$-Achse um die $x$- und $y$-Achse in Übereinstimmung zu bringen.
$$R_{x_i}(\alpha_i) = \left[ \begin{array}{cccc} 1 & 0 & 0 & 0 \\ 0 & \cos(\alpha_i) & - \sin(\alpha_i) & 0 \\ 0 & \sin(\alpha_i) & \cos(\alpha_i) & 0 \\ 0 & 0 & 0 & 1 \end{array} \right]$$
\end{enumerate}
In MAtrizenschreibweise lautet die Denavit-Hartenberg-Transformation von Armelement $i-1$ auf $i$:
\begin{eqnarray*}
A_{i-1,i} &=& R_{z_{i-1}}(\theta_i) \cdot T_{z_{i-1}}(s_i) \cdot T_{x_i}(a_i) \cdot R_{x_i}(\alpha_i) \\
&=& \left[ \begin{array}{cccc} \cos(\theta_i) & - \sin(\theta_i) \cdot \cos(\alpha_i) & \sin(\theta_i) \cdot \sin(\alpha_i) & a_i \cdot \cos(\theta_i) \\ \sin(\theta_i) & \cos(\theta_i) \cdot \cos(\alpha_i) & - \cos(\theta_i) \cdot \sin(\alpha_i) & a_i \cdot \sin(\theta_i) \\ 0 & \sin(\alpha_i) & \cos(\alpha_i) & s_i \\ 0 & 0 & 0 & 1 \end{array} \right]
\end{eqnarray*}
$\theta_i$ und $s_i$ sind variable Größen während der Bewegung des Roboters, abhängig von dessen spezieller Geometrie und Maßen. $\alpha_i$ und $a_i$ dagegen sind sowohl bei Rotations-, als auch bei Schubgelenken invariante Größen und müssen für die spätere Berechnung der direkten Kinematik nur einmal für jedes einzelne Armelement bestimmt werden.

\subsubsection*{Gelenktypen}

Um einen Roboter zu bewegen, sind Gelenke erforderlich (üblicherweise elektrisch angetrieben). Gelenktypen:
\begin{itemize}
\item Rotationsgelenk: Drehachse in $90\degree$ zu den Achsen der angeschlossenen Glieder.
\item Torsionsgelenk: Drehachse parallel zu den Achsen der angeschlossenen Glieder.
\item Revolvergelenk: Eingangsglied parallel, Ausgangsglied rechtwinklig zur Drehachse.
\item Lineargelenk: gleitende Bewegung entlang der Achse.
\end{itemize}

\subsection{Kinematik}

Zwei Probleme:
\begin{itemize}
\item Bei gegebenem Gelenkwinkel: Wo ist der Greifer? \\ $\to$ direktes kinematisches Problem
\item Bei gegebener Greifer-Sollposition: Wie müssen die Gelenkwinkel eingestellt werden? \\ $\to$ inverses kinematisches Problem
\end{itemize}

\subsubsection*{Direktes kinematisches Problem}

Um die Stellung des Greifers in Bezug auf das BKS zu bestimmen, werden die DH-Matrizen in der Reihenfolge der Armelemente multipliziert. \\ Bei $n$ Armelementen und $n$-dimensionalem Gelenkwinkelvektor $\theta$:
$$p_{TCP}(\theta) = \prod\limits_{i=1}^n A_{i-1,i}(\theta_i)$$

\subsubsection*{Inverses kinematisches Problem}

Aus den DH-Parametern und der Stellung des Greifers sollen die Gelenkwinkel bestimmt werden. Die Gleichung
$$p_{TCP}(\theta) = \prod\limits_{i=1}^n A_{i-1,i}(\theta_i)$$
muss daher nach $\theta$ aufgelöst werden. Für 6 Gelenke sind dies 12 Gleichungen mit 6 Unbekannten. Die Gleichungssysteme sind überbestimmt und damit existiert keine eindeutige Lösung.






