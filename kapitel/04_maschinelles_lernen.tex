% !TeX root = summary.tex

\section{Maschinelles lernen}

\subsection{Unsupervised Learning}

\begin{itemize}
\item Datensammlung und Bezeichnung kostenaufwendig und zeitverschwenderisch.
\item Die Charaktermerkmale der Muster können sich ändern im Verlauf der Zeit.
\item Vielleicht hat man keine Einsicht in die Struktur der Daten.
\end{itemize}
$\Rightarrow$ Die Klassen sind \underline{nicht} bekannt.

\subsection{Mischdichten}

\begin{itemize}
\item Beispiele kommen von $c$ Klassen
\item A priori Wahrscheinlichkeit $P(\omega_j)$
\item Annahme: Die Formen der Klassenbedingten PDF (probability density function -- Wahrscheinlichkeitsdichtefunktion) $P(X | \omega_j , \theta_j)$ sind bekannt
\item Unbekannter Parametervektor $\theta_1, \dots, \theta_c$
\end{itemize}
Problem: Haarige Mathematik \\ Vereinfachung / Approximation \\ Betrachte nur Mittel $\to$ Isodaten
\begin{itemize}
\item Wähle initiale $\mu_1, \dots, \mu_c$
\item Klassifiziere $n$ Muster zu dem Mittel, das am nähesten liegt.
\item Berechne die Mittel neu aus den Mustern der Klasse
\item Mittel hat sich geändert? Gehe zu Schritt 2, sonst: stopp
\end{itemize}
Isodaten, Probleme:
\begin{itemize}
\item Die Wahl der initialen Mittel $\mu$.
\item Wissen über die Anzahl der Klassen.
\item Annahme über die Verteilung.
\item Was bedeutet "{}nah"{}?
\end{itemize}

\subsection{Clustering}

\begin{itemize}
\item Ähnlichkeit
\item Kriteriumsfunktion
\item Muster in der selben Klasse sollten die Kriteriumsfunktion noch extremer machen, dies erfasst die Cluserqualität.
\item Beispiel: Summe des Fehlerkriteriums: $$J = \sum\limits_{i=1}^c \sum\limits_{\max} || x - m_i ||^2$$
\end{itemize}

\subsection{Hierarchisches Clustering}

\begin{itemize}
\item $c$ muss nicht bestimmt werden
\item Mittel müssen nicht erraten werden
\item[(1)] Initialisiere $c := n$
\item[(2)] Finde die nähesten Paare von eindeutigen Clustern $X_i$ und $X_j$
\item[(3)] Vereinige sie und dekrementiere $c$
\item[(4)] Wenn $c < C_{stop}$: stoppen, sonst: gehe zu Schritt (2)
\end{itemize}

\section{Neuronale Netze}

\subsection{Wieso neuronale Netze?}

\begin{itemize}
\item Massiver Parallelismus
\item Massive Randbedingungsgenugtuung für "{}krank"{}-definierte Eingaben
\item Einfache Recheneinheiten
\item Viele Prozesseinheiten, viele Verbindungen
\item Einheitlichkeit ($\to$ Sensorverschmelzung)
\item Nichtlineare Klassifizierer / Abbilder ($\to$ gute Performance)
\item Lernfähig / anpassbar
\end{itemize}

\subsection{Entscheidungsfunktion $g(x)$}

\begin{eqnarray*}
g(\myvector{x}) > 0 &\Rightarrow& \textrm{Klasse A} \\ g(\myvector{x}) < 0 &\Rightarrow& \textrm{nicht Klasse A} \\ g(\myvector{x}) = 0 &\Rightarrow& \textrm{keine Entscheidung}
\end{eqnarray*}
$$g(\myvector{x}) = \sum\limits_{i=1}^n w_1x_1 + w_0 = \myvector{w}^T \myvector{x} + w_0$$
$\myvector{x} = (x_1,\dots,x_n)^T$ : Eigenschaftsvektor, $\myvector{w} = (w_1,\dots,w_n)^T$ : Gewichtungsvektor, $w_0$ : Schwellenwert

\subsection{Lineare Diskriminantenfunktion}

Hyperebe $H$: $g(\myvector{x}) = \sum\limits_{i=1}^n w_1x_1 + w_0 = \myvector{w}^T \myvector{x} + w_0 = 0$
$$\Rightarrow \myvector{x} = q \frac{\myvector{w}}{|| \myvector{w} ||} + r \frac{\myvector{w}}{|| \myvector{w} ||} + \myvector{x}_p$$
Vektor $q \frac{\myvector{w}}{|| \myvector{w} ||}$ entspricht: $g \left(q \frac{\myvector{w}}{|| \myvector{w} ||} \right) = 0 = q ||\myvector{w}|| + w_0$ $$\Rightarrow q = - \frac{w_0}{|| \myvector{w} ||}$$
Und mit $g(\myvector{x}) = \myvector{w}^T q \frac{\myvector{w}}{|| \myvector{w} ||} + \myvector{w}^T r \frac{\myvector{w}}{|| \myvector{w} ||} + \myvector{w} \myvector{x}_p + w_0 = -w_0 + r ||\myvector{w}|| + w_0$ erhalten wir $$r = \frac{g(\myvector{x})}{|| \myvector{w} ||}$$

\subsection{Fisher-lineare Diskriminante}

\begin{itemize}
\item Dimensionsreduktion
\item Projeziert eine Menge von mehrdimensionalen Punkten auf eine Line $y = \myvector{w} \myvector{x}$
\item Die Fisher Diskriminaten ist eine Funktion, die folgendes Kriterium maximiert $$g(x) = \frac{| \tilde{m}_1 - \tilde{m_2}|}{\tilde{s}_1 + \tilde{s}_2}$$
wobei $\tilde{m}_i = \frac{1}{n} \sum\limits_{y \in Y_i} y$ : Mittel für projezierte Muster, $\tilde{s}_i^2 = \sum\limits_{y \in Y_i} (y - \tilde{m}_i)^2$ : Streuung für projezierte Muster
\item Fisher's lineare Diskriminante:
\begin{eqnarray*}
\myvector{w} &=& s_w^{-1} (\myvector{m}_1 - \myvector{m}_2) \\ s_w &=& s_1 + s_2 \\ s_i &=& \sum\limits_{x \in X_i} (\myvector{x} - \myvector{m}_i)(\myvector{x} - \myvector{m}_i)^T
\end{eqnarray*}
\end{itemize}

\subsection{Das Perzeptron}

\mypic{10}{perzeptron}

\subsubsection*{Lineare Diskriminantenfunktionen}

\begin{itemize}
\item Keine Annahme über die Verteilung (Nicht-parametrisch)
\item Lineare Entscheidungsflächen
\item Start durch überwachtes Training (Klassen der Trainingsdaten gegeben)
\item Diskriminantenfunktion: $$g(x) = w_0 + \sum\limits_{i=1}^n w_ix_i = \sum\limits_{i=0}^n w_ix_i, \quad x_0 = 1$$
\item $g(x)$ ergibt die Distanz von der Entscheidungsfläche
\item Zwei Kategorien Fall:
\begin{eqnarray*}
g_1(x) > 0 &\Rightarrow& \textrm{Klasse 1} \\ g_1(x) < 0 &\Rightarrow& \textrm{Klasse 2}
\end{eqnarray*}
\end{itemize}

\subsubsection*{Perzeptron}

$$g(x) = \sum\limits_{i=0}^n w_i x_i \quad x_0 = 1 \qquad \to \qquad \textrm{finde } w$$
Alle Vektoren $x_i$ sind korrekt benannt
\begin{eqnarray*}
\myvector{w}\myvector{x}_i > 0 && \textrm{ wenn } x_i \textrm{ zu } \omega_1 \textrm{ gehört} \\ \myvector{w}\myvector{x}_i < 0 && \textrm{ wenn } x_i \textrm{ zu } \omega_2 \textrm{ gehört}
\end{eqnarray*}
Oder setze alle Muster, die zu $\omega_2$ gehören auf ihr negatives ($-\myvector{x}_i$). Dann sind alle Vektoren korrekt klassifiziert, wenn $\myvector{w} \myvector{x}_i > 0$ für alle $i$.
\subsubsection*{Kriteriumsfunktion vom Perzeptron}
$$J_P(\myvector{w}) = \sum\limits_{x \in X} (- \myvector{w} \myvector{x})$$
$X$ ist die Menge der falsch klassifizierten Merkmale. Da $\myvector{w} \myvector{x}$ die Negation für falsch klassifizierte Merkmale ist, ist $J_P(\myvector{w})$ positiv. Sobald $J_P = 0$ gilt, ist ein Lösungsvektor gefunden. \\ $J_P$ ist proportional zu der Summe der Distanzen der falsch klassifizierten Mustern zur Entscheidungsgrenze.
$$\nabla J_P = \sum\limits_{x \in X} (- \myvector{x}) \qquad \myvector{w}_{k+1} = \myvector{w}_k + \zeta_k \sum\limits_{x \in X} \myvector{x}$$

\subsubsection*{Lernen des Perzeptron}

Fragen:
\begin{itemize}
\item Wie soll man die Lernrate setzen?
\item Wie soll man die initialen Gewichte setzen?
\end{itemize}
Probleme:
\begin{itemize}
\item Untrennbare Daten
\item Trennbare Daten, aber welche Entscheidungsfläche
\item Nicht linear trennbar
\end{itemize}

\subsubsection*{Variationen}

Entspannungsprozedur: $$J_q(\myvector{w}) = \sum\limits_{x \in X} (\myvector{w}^t \myvector{y})^2$$
Der Gradient ist kontinuierlich $\to$ gleichmäßige Fläche. Manchmal geht es gegen Null!
$$J_q(\myvector{w}) = \frac{1}{2} \sum\limits_{x \in X} \frac{(\myvector{x}^t\myvector{w} - b)^2}{||x||^2} \textrm{ begrenzt durch } b$$

\subsection{Generalisierung}

\begin{itemize}
\item Leistung auf Trainingsdaten interessiert uns nicht, sondern ungesehene real Welt\dots
\item Wie gut funktioniert mein System in unvorhergesehenen Situationen?
\end{itemize}

\subsubsection*{Generalisierungsfähigkeit}

Fähigkeit, das aus den Trainingsdaten Erlernte auf neue Daten (Testdaten) anzuwenden.

\subsubsection*{Drei Gründe für schlechte Generalisierung}
\begin{enumerate}
\item Overfitting / Overtraining (zu lange trainiert)
\item Zu viele Parameter (weights) bzw. zu wenig Trainingsmaterial
\item falsche Netzwerkstruktur
\end{enumerate}
\subsubsection*{Methoden zur Verbesserung der Generalisierung}

\begin{enumerate}
\item Training am Besten Punkt abbrechen
\begin{itemize}
\item Aufteilung der Daten in: Trainings-, Crossvalidation- und Test-Menge
\item Wichtig: Testdaten von Trainingsdaten trennen! Wiederholtes Testen ist leichtes Trainieren (Tuning) $\to$ Crossvalidation Set
\end{itemize}
\item Reduzierung der Komplexität des Netzwerkes durch Regularisierung: Weight Decay, Weight Elimination, Optimal Brain Damage, Optimal Brain Surgeon
\item Schrittweises Vergrößern eines zu kleinen Netzes (konstruktiv): Cascade Correlation, Meiosis Netzwerke, ASO (Automativ Structure Optimization)
\end{enumerate}














